\chapter{Dnevnik promjena dokumentacije}
		
		\textbf{\textit{Kontinuirano osvježavanje}}\\
				
		
		\begin{longtblr}[
				label=none
			]{
				width = \textwidth, 
				colspec={|X[2]|X[13]|X[3]|X[3]|}, 
				rowhead = 1
			}
			\hline
			\textbf{Rev.}	& \textbf{Opis promjene/dodatka} & \textbf{Autori} & \textbf{Datum}\\[3pt] \hline
			0.1 & Napravljen predložak.	& Agejev & 30.10.2023. 		\\[3pt] \hline
			0.2 & Dodani obrasci uporabe, dijagrami obrazaca uporabe, akteri i dionici. & Skukan & 29.10.2023. \\[3pt] \hline 
			0.3	& Opis projektnog zadatka & Agejev, Haralović & 02.11.2023. 	\\[3pt] \hline
			0.4	& Dijagram razreda & Vidović & 03.11.2023. 	\\[3pt] \hline
			0.5	& Opis baze podataka & Tomić & 05.11.2023. 	\\[3pt] \hline
			0.6 & Prva inačica sekvencijskih dijagrama & Lovrinović & 6.11.2023. \\[3pt] \hline
			0.7 & Ostali zahtjevi & Agejev & 7.11.2023. \\[3pt] \hline
			0.8 & Prepravljeni sekvencijski dijagrami. \newline Dodano opcionalno filtriranje. & Skukan & 9.11.2023. \\[3pt] \hline
			0.9 & Prepravljeni obrasci uporabe i dijagrami. \newline Maknut UC17. & Skukan & 14.11.2023. \\[3pt] \hline
			0.10 & Ispravci kod aktera i obrazaca. \newline Nova slika dijagrama obrazaca. & Skukan & 16.11.2023. \\[3pt] \hline
			0.11 & Priprema za predaju prve inačice. & Agejev & 17.11.2023. \\[3pt] \hline
			& & \\[3pt] \hline
			
		\end{longtblr}
	
	
		\textit{Moraju postojati glavne revizije dokumenata 1.0 i 2.0 na kraju prvog i drugog ciklusa. Između tih revizija mogu postojati manje revizije već prema tome kako se dokument bude nadopunjavao. Očekuje se da nakon svake značajnije promjene (dodatka, izmjene, uklanjanja dijelova teksta i popratnih grafičkih sadržaja) dokumenta se to zabilježi kao revizija. Npr., revizije unutar prvog ciklusa će imati oznake 0.1, 0.2, …, 0.9, 0.10, 0.11.. sve do konačne revizije prvog ciklusa 1.0. U drugom ciklusu se nastavlja s revizijama 1.1, 1.2, itd.}