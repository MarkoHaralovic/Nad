\chapter{Opis projektnog zadatka}
		
		
		Namjera projekta riješiti je trenutačan manjak strane literature prevedene na hrvatski ili srodni jezik. Količina kvalitetne literature ograničena je i često teško dostupna u Hrvatskoj, dok web poslužitelji i stranice ne nude centralizirano rješenje za pronalazak domaćih izdanja niti jednostavnu opciju zahtjeva novih prijevoda. Naime, web stranice raznih antikvarijata nisu ažurne, metode pretraživanja nezgrapne su, a u katalogu imaju popis knjiga koji ne odgovara trenutačnom stanju u njihovim skladištima. Mnogo knjiga u njihovoj ponudi nije prevedena ili ne sadrži naslov izvornika čime je nabava željene knjige otežana.
		
		Svojim rješenjem nadamo se olakšati krajnjem korisniku pretragu i nabavku željene knjige, osobito domaćih izdanja. Naše rješenje temelji se na izradi web stranice koja služi kao posrednik između ponuditelja i korisnika, s ažurnom bazom knjiga koja brojčano nadmašuje ostale stranice. 
		
		Poseban naglasak stavljamo na literaturu prevedenu na hrvatski i jezike slične hrvatskom, a korisnicima pojednostavljujemo proces pronalaska i odabira željenog naslova dostupnošću raznovrsnih ponuditelja (izdavačkih kuća, preprodavača, antikvarijata).
		
		Web stranica služila bi kao središnje mjesto ponude različitih izdavačkih kuća, manjih knjižara i individualnih preprodavača, na taj način stvarajući veću ponudu, bolju pokrivenost i konkurentnije cijene krajnjem korisniku.
		
		Krajnjem korisniku olakšavamo pregled trenutno dostupnih knjiga na hrvatskom, srodnim i stranim jezicima. Nudimo lokacijsku preglednost ponude određenog naslova, odnosno informaciju o dostupnosti na stranom tržištu, ukoliko je knjiga nedostupna na lokalnom području.
		
		Također nudimo mogućnost zahtjeva prijevoda knjige ovisno o interesima korisnika, izdavač može ostvariti kontakt s izdavačem strane knjige i ishodovati dozvolu za prijevod knjige na jezik blizak krajnjem korisniku.\\
		
		
		
		\subsection{Slična rješenja}
		
		Trenutno je tržište knjiga u velikoj mjeri digitalizirano. Naravno, i dalje postoje knjižare te knjige se i dalje kupuju u fizičkom obliku, ali većina izdavačkih kuća nudi svoj asortiman na web stranicama. Mnoge izdavačke kuće prenijele su dio svoje ponude na web, što je jasan pokazatelj korisničkog interesa za sličnim rješenjima.
		
		Tako, na primjer, Naklada Znanje na svom portalu nudi knjige podijeljene u kategorije knjiga prevedenih na hrvatski te strane knjige, većinom na engleskom jeziku. Njihov asortiman nije ograničen samo na knjige; imaju i kategorije poput igračaka, multimedije i slično. Na slici 2.1 vidimo njihovu web stranicu.
		
		\begin{figure}[H]
			\includegraphics[scale=1]{slike/naklada-znanje.PNG} %veličina slike u odnosu na originalnu datoteku i pozicija slike
			\centering
			\caption{web stranica naklade \textit{Znanje}}
			\label{fig:Naklada Znanje}
		\end{figure}
		
		Na njihovoj stranici možete pretraživati knjige po naslovu, autoru ili ključnim riječima, a omogućena je i jednostavna kupovina.
		
		Eknjiga.hr nudi slično rješenje, s tom razlikom da na njihovoj stranici korisnik može odabrati i izdavača u sekciji "Izdavač", omogućavajući tako izdavačima i korisnicima da lakše pristupe ponudi knjiga. Takva ponuda sliči rješenju koje mi nudimo, odnosno centraliziranom i neovisnom posredniku između ponuditelja i korisnika.
		
		Na slici 2.2 vidimo njihovo rješenje:
		
		\begin{figure}[H]
			\includegraphics[scale=1]{slike/Eknjiga.PNG} %veličina slike u odnosu na originalnu datoteku i pozicija slike
			\centering
			\caption{web stranica \textit{Eknjiga.hr}}
			\label{fig:Eknjiga}
		\end{figure}
		
		
		Zbog sličnosti s gore navedenim, nećemo dodatno opisivati slične portale i stranice, ali ih ovdje navodimo: Mozaik knjiga, Školska knjiga, Super knjižara, Hoću knjigu, VBZ knjižara, Knjiga HR, Čitaj knjigu, Svijet knjige, Libristo, Profil knjiga, E-knjižara i mnogi drugi. Također, knjige se nude i na aplikacijama poput Njuškala.
		
		Naše rješenje razlikuje se po nekoliko aspekata. Većina postojećih platformi nudi kategorijsku podjelu knjiga, promotivne akcije i ponude usmjerene korisnicima koji ciljano pristupaju tim knjižarama kako bi pronašli određeni naslov ili se izložili širokoj ponudi te time možda pronašli knjigu od interesa. Taj pristup je usmjeren prema promociji cjelokupne ponude određenog izdavača ili nakladnika, s ciljem privlačenja korisnika na kupnju kod njih. 
		
		S druge strane, naša platforma nije posvećena jednom izdavaču, ali ne dozvoljava ni neprovjerene i privatne prodavače. Umjesto toga, dizajnirana je s korisnikom u središtu pažnje, omogućavajući mu da na temelju lokacije, jezika i naslova pronađe knjigu po najpovoljnijim uvjetima kupnje – bilo da je to cjenovna prihvatljivost, pogodna lokacija kupovine ili povjerenje u izdavača.
		
		\subsection{Ciljana publika}
		
		Naša stranica namijenjena je svima koji žele kupiti knjigu, a osobito onima koji ne znaju strane jezike ili radije čitaju na materinjem ili srodnom jeziku. Ponuditeljima stranica nudi uvid u želje kupaca putem skupljanja zahtjeva za prijevode te platformu za prodaju.
		
		Uzimajući u obzir korisnika koji ima preferira određenog izdavača u odnosu na ostale, teško je za očekivati kako će se isti odlučiti za promjenom i našim rješenjem ukoliko odabrani naslov koji traži može pronaći kod tog izdavača.
		
		Naša web stranica namijenjena je sljedećim tipovima kupaca:
		
		\begin{packed_enum}
			\item korisniku sa specifičnim zahtjevom knjige na preferiranom jeziku ili određenoj lokaciji, kojemu je u interesu pronaći knjigu koju možebitno nema u katalogu izdavača od povjerenja
			\item korisniku koji po lokaciji i odabranom jeziku želi pregledati ponudu knjiga koje bi ga mogle zanimati
			\item neodlučnom korisniku koji nije siguran u odabir knjige (cijena, udaljenost izdavača)
		\end{packed_enum}
		
		Registriranim ponuditeljima naša stranica nudi:	
		
		\begin{packed_enum}
			\item stranicu sa značajkama atraktivnim kupcima
			\item izbjegavanje potrebe vlastitog web rješenja
			\item mogućnošću prodaje knjiga koje su rijetko kupovane te su ispale iz glavne ponude
			\item uvid u želje korisnika za prijevodima
		\end{packed_enum}
		
		
		
		
		
		
		\section{Primjeri u \LaTeX u}
		
		\textit{Ovo potpoglavlje izbrisati.}\\

		U nastavku se nalaze različiti primjeri kako koristiti osnovne funkcionalnosti \LaTeX a koje su potrebne za izradu dokumentacije. Za dodatnu pomoć obratiti se asistentu na projektu ili potražiti upute na sljedećim web sjedištima:
		\begin{itemize}
			\item Upute za izradu diplomskog rada u \LaTeX u - \url{https://www.fer.unizg.hr/_download/repository/LaTeX-upute.pdf}
			\item \LaTeX\ projekt - \url{https://www.latex-project.org/help/}
			\item StackExchange za Tex - \url{https://tex.stackexchange.com/}\\
		
		\end{itemize} 	


		
		\noindent \underbar{podcrtani tekst}, \textbf{podebljani tekst}, 	\textit{nagnuti tekst}\\
		\noindent \normalsize primjer \large primjer \Large primjer \LARGE {primjer} \huge {primjer} \Huge primjer \normalsize
				
		\begin{packed_item}
			
			\item  primjer
			\item  primjer
			\item  primjer
			\item[] \begin{packed_enum}
				\item primjer
				\item[] \begin{packed_enum}
					\item[1.a] primjer
					\item[b] primjer
				\end{packed_enum}
				\item primjer
			\end{packed_enum}
			
		\end{packed_item}
		
		\noindent primjer url-a: \url{https://www.fer.unizg.hr/predmet/proinz/projekt}
		
		\noindent posebni znakovi: \# \$ \% \& \{ \} \_ 
		$|$ $<$ $>$ 
		\^{} 
		\~{} 
		$\backslash$ 
		
		
		\begin{longtblr}[
			label=none,
			entry=none
			]{
				width = \textwidth,
				colspec={|X[8,l]|X[8, l]|X[16, l]|}, 
				rowhead = 1,
			} %definicija širine tablice, širine stupaca, poravnanje i broja redaka naslova tablice
			\hline \SetCell[c=3]{c}{\textbf{naslov unutar tablice}}	 \\ \hline[3pt]
			\SetCell{LightGreen}IDKorisnik & INT	&  	Lorem ipsum dolor sit amet, consectetur adipiscing elit, sed do eiusmod  	\\ \hline
			korisnickoIme	& VARCHAR &   	\\ \hline 
			email & VARCHAR &   \\ \hline 
			ime & VARCHAR	&  		\\ \hline 
			\SetCell{LightBlue} primjer	& VARCHAR &   	\\ \hline 
		\end{longtblr}
		

		\begin{longtblr}[
				caption = {Naslov s referencom izvan tablice},
				entry = {Short Caption},
			]{
				width = \textwidth, 
				colspec = {|X[8,l]|X[8,l]|X[16,l]|}, 
				rowhead = 1,
			}
			\hline
			\SetCell{LightGreen}IDKorisnik & INT	&  	Lorem ipsum dolor sit amet, consectetur adipiscing elit, sed do eiusmod  	\\ \hline
			korisnickoIme	& VARCHAR &   	\\ \hline 
			email & VARCHAR &   \\ \hline 
			ime & VARCHAR	&  		\\ \hline 
			\SetCell{LightBlue} primjer	& VARCHAR &   	\\ \hline 
		\end{longtblr}
	


		
		
		%unos slike
		\begin{figure}[H]
			\includegraphics[scale=0.4]{slike/aktivnost.PNG} %veličina slike u odnosu na originalnu datoteku i pozicija slike
			\centering
			\caption{Primjer slike s potpisom}
			\label{fig:promjene}
		\end{figure}
		
		\begin{figure}[H]
			\includegraphics[width=\textwidth]{slike/aktivnost.PNG} %veličina u odnosu na širinu linije
			\caption{Primjer slike s potpisom 2}
			\label{fig:promjene2} %label mora biti drugaciji za svaku sliku
		\end{figure}
		
		Referenciranje slike \ref{fig:promjene2} u tekstu.
		
		\eject
		
	