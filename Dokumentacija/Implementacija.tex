\chapter{Implementacija i korisničko sučelje}
		
		
		\section{Korištene tehnologije i alati}
		
			
			 
			 Za timsku komunikaciju koristili smo WhatsApp\footnote{https://www.whatsapp.com/} koji nam je pružio brz i jednostavan medij. Za vizualizaciju i modeliranje sustava, Astah Professional\footnote{http://astah.net/editions/professional} bio je naš odabir, omogućio nam je izradu UML dijagrama koji su olakšali razumijevanje strukture i funkcionalnosti projekta.\\
			 
			 Git\footnote{https://git-scm.com/} je bio ključan u procesu upravljanja izvornim kodom, omogućavajući nam upravljanje inačicama i suradnju na projektu. Udruženi repozitorij projekta smješten na GitHub\footnote{https://github.com} platformi pružio je centralizirano mjesto za pohranu, pregled i praćenje promjena u kodu.\\
			 
			 Za razvoj softvera koristili smo Rider\footnote{https://www.jetbrains.com/rider/} integrirano razvojno okruženje (IDE) koje nam je omogućilo učinkovitiju izradu aplikacije.\\
			 
			 Na strani backenda, odabrali smo ASP.NET Core radni okvir\footnote{https://dotnet.microsoft.com/en-us/apps/aspnet/} u kombinaciji s jezikom C\#\footnote{https://docs.microsoft.com/en-us/dotnet/csharp/}, a za frontend smo se uz standardni HTML,\footnote{https://www.w3.org/html/} CSS\footnote{https://www.w3.org/Style/CSS/Overview.en.html} i JS\footnote{https://www.javascript.com/} oslonili na htmx\footnote{https://htmx.org/} radi njegove jednostavnosti.\\
			 
			 Baza podataka implementirana u PostgreSQL-u\footnote{https://www.postgresql.org/} smještena je na poslužitelju u oblaku Microsoft Azure\footnote{https://portal.azure.com/}.\newline
			
			
			\eject 
		
	
		\section{Ispitivanje programskog rješenja}
		
			Opsežno ispitivanje nažalost nismo napravili, no testirali smo programsko rješenje ručno.
			
			
			\eject 
		
		
		\section{Dijagram razmještaja}
			
			Dijagram razmještaja prikazuje topologiju sklopovlja i njegove programske potpore koji služe za implementaciju sustava u njegovom radnom okruženju. Na poslužiteljskom računalu se nalaze web poslužitelj i poslužitelj baze podataka. Korisnik se mora služiti web preglednikom da može pristupiti web aplikaciji. Sustav je baziran na arhitekturi "klijent - poslužitelj". Komunikacija između računala korisnika i poslužitelja se uspostavlja pomoću HTTP veze
			 \begin{figure}[hbt!]
			 	\centering
			 	\includegraphics[width = \textwidth]{slike/Dijagram razmještaja}
			 	\caption{Slika dijagrama razmještaja}
			 	\label{fig:razmjestaj}
			 \end{figure}
			
			\eject 
		
		\section{Upute za puštanje u pogon}
		
			\textbf{\textit{dio 2. revizije}}\\
		
			 \textit{U ovom poglavlju potrebno je dati upute za puštanje u pogon (engl. deployment) ostvarene aplikacije. Na primjer, za web aplikacije, opisati postupak kojim se od izvornog kôda dolazi do potpuno postavljene baze podataka i poslužitelja koji odgovara na upite korisnika. Za mobilnu aplikaciju, postupak kojim se aplikacija izgradi, te postavi na neku od trgovina. Za stolnu (engl. desktop) aplikaciju, postupak kojim se aplikacija instalira na računalo. Ukoliko mobilne i stolne aplikacije komuniciraju s poslužiteljem i/ili bazom podataka, opisati i postupak njihovog postavljanja. Pri izradi uputa preporučuje se \textbf{naglasiti korake instalacije uporabom natuknica} te koristiti što je više moguće \textbf{slike ekrana} (engl. screenshots) kako bi upute bile jasne i jednostavne za slijediti.}
			
			
			 \textit{Dovršenu aplikaciju potrebno je pokrenuti na javno dostupnom poslužitelju. Studentima se preporuča korištenje neke od sljedećih besplatnih usluga: \href{https://aws.amazon.com/}{Amazon AWS}, \href{https://azure.microsoft.com/en-us/}{Microsoft Azure} ili \href{https://www.heroku.com/}{Heroku}. Mobilne aplikacije trebaju biti objavljene na F-Droid, Google Play ili Amazon App trgovini.}
			
			
			\eject 